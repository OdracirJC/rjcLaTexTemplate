\documentclass{article}



\usepackage{amsmath, amssymb, amsthm}
\usepackage{graphicx}
\usepackage{lipsum}
\usepackage{varioref}
\usepackage{listings}
\usepackage{color}

\definecolor{mygreen}{rgb}{0,0.6,0}
\definecolor{mygray}{rgb}{.95,.95,.95}
\definecolor{mymauve}{rgb}{0.58,0,0,82}

\lstset{
  backgroundcolor=\color{mygray},   % choose the background color; you must add \usepackage{color} or \usepackage{xcolor}; should come as last argument
  basicstyle=\small,        % the size of the fonts that are used for the code
  breakatwhitespace=false,         % sets if automatic breaks should only happen at whitespace
  breaklines=true,                 % sets automatic line breaking
  captionpos=t,                    % sets the caption-position to bottom
  commentstyle=\color{mygreen},    % comment style
  extendedchars=true,              % lets you use non-ASCII characters; for 8-bits encodings only, does not work with UTF-8
  firstnumber=1,                % start line enumeration with line 1000
  frame=tb,	                   % adds a frame around the code
  keepspaces=true,                 % keeps spaces in text, useful for keeping indentation of code (possibly needs columns=flexible)
  keywordstyle=\color{blue},       % keyword style
  language=C++,                 % the language of the code
  numbers=left,                    % where to put the line-numbers; possible values are (none, left, right)
  numbersep=5pt,                   % how far the line-numbers are from the code
  rulecolor=\color{black},         % if not set, the frame-color may be changed on line-breaks within not-black text (e.g. comments (green here))
  showspaces=false,                % show spaces everywhere adding particular underscores; it overrides 'showstringspaces'
  showstringspaces=false,          % underline spaces within strings only
  showtabs=false,                  % show tabs within strings adding particular underscores
  stepnumber=1,                    % the step between two line-numbers. If it's 1, each line will be numbered
  stringstyle=\color{mymauve},     % string literal style
  tabsize=2,	                   % sets default tabsize to 2 spaces
  title=\lstname,                   % show the filename of files included with \lstinputlisting; also try caption instead of title
}

\theoremstyle{definition}
\newtheorem{definition}{Definition}

\theoremstyle{plain}
\newtheorem{theorem}{Theorem}
\theoremstyle{remark}



\title{Title}
\date{04/04/2000}
\author{Ricardo J. Cantarero}


\begin{document}

\maketitle

\tableofcontents

\listoffigures

\listoftables

\section*{Abstract}

\newpage


\section{Introduction}

\lipsum[1-3]
Math $cos\pi=-1$.

\section{Methods}

\lipsum[5]

\subsection{Paragraphs}
\lipsum[6]
\paragraph{Paragraph Descrption} \lipsum[7]

\paragraph{Paragraph Descrption} \lipsum[8]

\subsection{Math}
\lipsum[4]

\begin{equation}
\cos^{3}\theta = \frac{1}{4}\cos\theta+\frac{3}{4}\cos 3\theta
\label{eq:refname2}
\end{equation}

\lipsum[5]

\begin{definition}[Gauss]
$\int_{-\infty}^{+\infty}
e^{-x^2}\,dx=\sqrt{\pi}$.
\end{definition}
\begin{theorem}[Pythagoras]
The square of the hypotenuse (the side opposite the right angle) is equal to the sum of the squares of the other two sides.
\end{theorem}
\begin{proof}
We have that $\log(1)^2 = 2\log(1)$.
But we also have that $\log(-1)^2=\log(1)=0$.
Then $2\log(-1)=0$, from which the proof.
\end{proof}
\section{Some Code Examples}
\subsection{C++ Code}
\begin{lstlisting}
#include <iostream>
  void foo(int &);
  void bar(int);
  auto main() -> int // Suffix return type
  {
    int a {27}; //Modern C++ initialization
    int b {27};

    foo(a); // Pass by Ref
    bar(b); // Pass by Val

    std::cout << a << std::endl;
    std::cout << b << std::endl;

    return 0;
  }
  void foo(int& ref)
  {
    ref++;
  }
  void bar(int val)
  {
    val++;
  }
\end{lstlisting}
\newpage
\section{Images}

\end{document}
